\documentclass[12pt,a4paper,bibliography=totocnumbered,listof=totocnumbered]{scrartcl}
\usepackage[ngerman]{babel}
\usepackage[utf8]{inputenc}
\usepackage{amsmath}
\usepackage{amsfonts}
\usepackage{amssymb}
\usepackage{graphicx}
\usepackage{fancyhdr}
\usepackage{tabularx}
\usepackage{geometry}
\usepackage{setspace}
\usepackage[right]{eurosym}
\usepackage[printonlyused]{acronym}
\usepackage{subfig}
\usepackage{floatflt}
\usepackage[usenames,dvipsnames]{color}
\usepackage{colortbl}
\usepackage{paralist}
\usepackage{array}
\usepackage{titlesec}
\usepackage{parskip}
\usepackage[right]{eurosym}
\usepackage{colortbl}
\usepackage[subfigure,titles]{tocloft}
\usepackage[pdfpagelabels=true]{hyperref}
\usepackage[frenchb]{babel}
\usepackage[round]{natbib}
\usepackage[utf8]{inputenc}
\usepackage{listings}

\lstset{basicstyle=\footnotesize, captionpos=b, breaklines=true, showstringspaces=false, tabsize=2, frame=lines, numbers=left, numberstyle=\tiny, xleftmargin=2em, framexleftmargin=2em}
\makeatletter
\def\l@lstlisting#1#2{\@dottedtocline{1}{0em}{1em}{\hspace{1,5em} Lst. #1}{#2}}
\makeatother

\geometry{a4paper, top=27mm, left=30mm, right=20mm, bottom=35mm, headsep=10mm, footskip=12mm}

\hypersetup{unicode=false, pdftoolbar=true, pdfmenubar=true, pdffitwindow=false, pdfstartview={FitH},
	pdftitle={Bachelorarbeit},
	pdfauthor={Alina Asisof},
	pdfsubject={Bachelorarbeit},
	pdfcreator={\LaTeX\ with package \flqq hyperref\frqq},
	pdfproducer={pdfTeX \the\pdftexversion.\pdftexrevision},
	pdfkeywords={Bachelorarbeit},
	pdfnewwindow=true,}

\pdfinfo{/CreationDate (D:20110620133321)}


\begin{document}

\titlespacing{\section}{0pt}{12pt plus 4pt minus 2pt}{-6pt plus 2pt minus 2pt}

% Kopf- und Fusszeile
\renewcommand{\sectionmark}[1]{\markright{#1}}
\renewcommand{\leftmark}{\rightmark}
\pagestyle{fancy}
\lhead{ }
\chead{ }
\rhead{\thesection\space\contentsname}
\lfoot{Software Engineering}
\cfoot{ }
\rfoot{\ \linebreak Seite \thepage}
\renewcommand{\headrulewidth}{0.4pt}
\renewcommand{\footrulewidth}{0.4pt}

% Vorspann
\renewcommand{\thesection}{\Roman{section}}
\renewcommand{\theHsection}{\Roman{section}}
\pagenumbering{Roman}

% ----------------------------------------------------------------------------------------------------------
% Titelseite
% ----------------------------------------------------------------------------------------------------------
\thispagestyle{empty}
\begin{center}
	
	\vspace*{2cm}
	\Huge
	\textbf{Software Engineering}\\
\vfill
	\Large
	\textbf{The effect of the abandonment of the Swiss Franc - Euro parity on tourism}\\
\vfill
	\Large
	Prof. Dr. Philipp Zahn}\\
	\Large
	University of St. Gallen
	
\vfill


\end{center}

\begin{flushleft}	
	\vfill
	\normalsize
\textbf{Submitted by:} \\
Alina Asisof, Master of Business Innovation HSG\\
Student Number: 12-749-677\\ 
Carla Walker, Master of Business Innovation HSG\\ 
Student Number: xx-xxx-xx\\ 
Joao Matias, Bachelor of XYZ University of Lissabon\\ 
Student Number: xx-xxx-xxx\\ 
\vfill
Group Project Documentation\\ 
 
\vfill
\textbf{18. January 2017}

\end{flushleft}	



\pagebreak

% ----------------------------------------------------------------------------------------------------------
% Verzeichnisse
% ----------------------------------------------------------------------------------------------------------
% TODO Typ vor Nummer

\renewcommand{\cftfigpresnum}{Abb. }
\settowidth{\cfttabnumwidth}{Abb. 10\quad}
\settowidth{\cftfignumwidth}{Abb. 10\quad}

\titlespacing{\section}{0pt}{12pt plus 4pt minus 2pt}{2pt plus 2pt minus 2pt}
\singlespacing
\rhead{CONTENT}
\renewcommand{\contentsname}{Content}

\phantomsection
\addcontentsline{toc}{section}{\texorpdfstring{\hspace{0.35em}Content}{Content}}
\addtocounter{section}{1}
\tableofcontents
\pagebreak

%\pagebreak


\pagebreak


% ----------------------------------------------------------------------------------------------------------
% Inhalt
% ----------------------------------------------------------------------------------------------------------
% Abstände Überschrift
\titlespacing{\section}{0pt}{12pt plus 4pt minus 2pt}{-6pt plus 2pt minus 2pt}
\titlespacing{\subsection}{0pt}{12pt plus 4pt minus 2pt}{-6pt plus 2pt minus 2pt}
\titlespacing{\subsubsection}{0pt}{12pt plus 4pt minus 2pt}{-6pt plus 2pt minus 2pt}

% Kopfzeile
\renewcommand{\sectionmark}[1]{\markright{#1}}
\renewcommand{\subsectionmark}[1]{ }
\renewcommand{\subsubsectionmark}[1]{ }
\lhead{Chapter \thesection}
\rhead{\rightmark}

\onehalfspacing
\renewcommand{\thesection}{\arabic{section}}
\renewcommand{\theHsection}{\arabic{section}}
\setcounter{section}{0}
\pagenumbering{arabic}
\setcounter{page}{1}

%--------------------
%Einleitung
%--------------------

\section{Introduction}

%What problem does our project solve ? 
%For what will our code be used? 
%Research question: What effect had the abandonment of the Swiss Franc - Euro parity of January 2015 on tourism within and outside of Switzerland? 
%How does the strong CHF influence travel behavior? 
%CHF goes up - people travel more outside of Switzerland? 
%How will we analyze this? 
%What are the tools that we are going to use? Python and Latex. 

On January 15th the Swiss National Bank decided to abandon the Swiss Franc - Euro parity of 1.20:1. The financial markets reacted immediatly: The exchange rate fell drastically and a Euro temporarily was worth less than 0.8 Swiss Francs. The decision surprised experts and executives alike and evoked great concerns regarding the future competitiveness of the Swiss industries. Some of the analysis are yet to be made and conclusions to be drawn. The risen value of the Swiss currency also left the national tourism industry sorrowful, as hotel owners and owners of popular touristic attractions alike feared a diminishing number of foreign tourists and hence missing out on revenue. Our project will focus on the tourism industry: We will conduct a simple regression to see, whether the decision of the Swiss National Bank on abandoning the parity had any effect on the national tourism industry. Did the tourism in the country decline or did it remain unchanged? How did the strong Swiss Franc influence the travel behaviour within and outside of Switzerland? Are the Swiss travelling more since January 15th due to the risen value of their currency? Can we conclude an impact on tourism on the basis of tourist numbers? 


We will analyse this by using a data set provided by the Federal Statistical Office. The Federal Statistical Office publishes a variety of data on tourism throughout the year. For 2015 there is a comprehensive set of data available, gathered in form of Excel-files and officialy published. For the year 2016 there is a data set available gathering the data of the first half of the year which we will use as a proxy for the year 2016, since more recent data will not be available before February. These data sets will be compared to the data before January 15th/2015 and a regression on the exchange rate of the Swiss Franc to Euro will be made. We will therefore use financial data published at "Finanzen.net", so both sources are trustworthy and well known. We will compute a regression with the help of a code written in Python on Spyder. To do that, we will remotely collaborate with the help of GitHub, a platform used to write code in groups. GitHub will also help us with the coordination and collaboration challenge connected to our project.\\
For the documentation of our project we have used LateX, as it is a tool easily embedding both Python Code and different types of content. \\
\\
We will first introduce our data set and parameters. In the following chapter we will conduct the regression and present the results of our Python code. We will add a short explanation on the code, so that the code can be easily understood and reproduced. In the last part we will conclude on the tourism effects of the parity abandonement and assess, to what extend the concerns expressed by hotel owners and experts were justified. 

\newpage


% ----------------------------------------------------------------------------------------------------------
% 
% ----------------------------------------------------------------------------------------------------------
\section{Dataset}
The data set used for the regression 
BFS, Fianzen.net

\newpage
\section{Regression}

%Lorem ipsum
 

%\subsection{something}
\newpage
\section{Conclusion}
\section{Work progress summary}
%
%----------------------------------------------------------------------------------------------------------

% ----------------------------------------------------------------------------------------------------------


%\section{blabla} ----------------------------------------------------------------------------------------------------------

% ----------------------------------------------------------------------------------------------------------
% Literatur
% ----------------------------------------------------------------------------------------------------------

% Literaturliste soll im Inhaltsverzeichnis auftauchen


%\bibliographystyle{eehd_url}


% Literaturliste endgültig anzeigen
%\bibliography{literatur}

%-------------------


% ----------------------------------------------------------------------------------------------------------
% Anhang
% ----------------------------------------------------------------------------------------------------------

\section{Appendix}



\subsection{Tables}

\subsection{Statistics}





\end{document}




